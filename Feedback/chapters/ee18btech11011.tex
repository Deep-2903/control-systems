\begin{enumerate}[label=\thesection.\arabic*.,ref=\thesection.\theenumi]
\numberwithin{equation}{enumi}

\item The circuit in Fig. \ref{fig:ee18btech11011_original_circuit} utilizes a voltage amplifier with gain $\mu$ in a shunt-shunt feedback topology with the feedback network composed of resistor $R_F$, to use the feedback equations convert the signal source to its Norton Representation.Also draw the H-Circuit and G-Circuit. 
\begin{enumerate}
\item If the loop gain is very large, what approximate closed-loop voltage gain $V_{o}/V_{s}$ is realized?Also if $R_{s}$ = 1 k$\ohm$ , give the value of $R_{F}$ that will result in $V_{o}/V_{s}$ \simeq -10 V/V.

\item If the amplifier $\mu$ has a dc gain of $10^3$ V/V, an input resistance $R_{i d}$ = 100 k$\ohm$ , and an output resistance $r_{o}$ = 1 k$\ohm$ , find the actual $V_{o}/V_{s}$ realized. Also find $R_{i n}$ and $R_{o u t}$.

\item If the amplifier $\mu$ has an upper 3-dB frequency of 1 kHz and a uniform -20-dB/decade gain rolloff, what is the 3-dB frequency of the gain \mid$V_{o}/V_{s}$\mid.
\end{enumerate}
\renewcommand{\thefigure}{\theenumi.\arabic{figure}}
%
\begin{figure}[!ht]
	\begin{center}
		
		\resizebox{\columnwidth}{!}{\begin{circuitikz}[american voltages]
\ctikzset{bipoles/length=1cm}

\draw 
(0, 0) node[op amp] (opamp) {}
(opamp.-) to [short , *-] (-0.6,0.35) to[R =$R_{s}$] (-2.6,0.35) to[V=$V_{s}$] (-2.6,-2.05) node[ground]{}
(opamp.-) --(-0.85,1) to[R=$R_{F}$] (1,1) to [short, -*] (1,0) to [short, -o](2,0) node at(2.3,0){$V_{o}$}
(opamp.out) to (1,0) 
(opamp.+) -- (-0.85,-0.35) -- (-0.85,-1) node[ground]{};
\draw
(-1.3,-1.9) -- (-1.3,-0.2) to [short, i_>=$ $] (-0.8,-0.2);
\draw
node at (-1.3,-2.05){$R_{i n}$}
node at (0,0){$\mu$}
node at (2.3,-2.05){$R_{o u t}$};
\draw
(2.3,-1.9) -- (2.3,-0.2) to [short, i_>=$ $] (1.5,-0.2);
\end{circuitikz}

}
	\end{center}
\caption{}
\label{fig:ee18btech11011_original_circuit}
\end{figure}
%
\\
\solution  See Fig. \ref{fig:ee18btech11011_Norton_Representation} for the Norton Representation, Fig. \ref{fig:ee18btech11011_beta_circuit} for the H-Circuit , Fig.\ref{fig:ee18btech11011_A1_circuit} and Fig. \ref{fig:ee18btech11011_A2_circuit} for the G-Circuit. 
%
\begin{figure}[!ht]
	\begin{center}
			\resizebox{\columnwidth}{!}{\begin{circuitikz}[american]
\usetikzlibrary{positioning, fit, calc}
\draw (0,0) to[I = $I_{s}$] (0,2) -- (2,2) to[R=$R_{s}$,*-*] (2,0){}
(2,2) to [short, -o] (2.5,2) -- (5,2) {}
(7,1)node[draw,minimum width=4cm,minimum height=4cm] (load) {Gain Amplifier}{}
(7,-4)node[draw,minimum width=4cm,minimum height=4cm] (load) {Feedback Network}{}
(0,0) to [short, -o] (2.5,0) -- (5,0)
(13,0) to [R=$R_{L}$,*-*] (13,2) to [short, -o] (12,2) to[short, i = $I_{o}$] (9,2)
(3,0) to [short, *-] (3,-5) -- (5,-5){}
(14,2) to [short, o-] (13,2){}
(4,2) to [short, *-] (4,-3) -- (5,-3){}
(10,2) to [short, *-] (10,-3) -- (9,-3){}
(9,0) to [short, -*] (11,0) to [short, -o] (12,0){}
(9,-5) -- (11,-5) -- (11,0) to [short, -o] (14,0){}
;
\draw
node at (14,0.2){$-$}
node at (14,1.8){$+$}
node at (14.3,1){$V_{o}$};
\end{circuitikz}
}
	\end{center}
\caption{Shunt Shunt Amplifier Block Diagram}
\label{fig:Shunt_Shunt_Amplifier_Block_Diagram}
\end{figure}
%
\begin{figure}[!ht]
	\begin{center}
			\resizebox{\columnwidth}{!}{\begin{circuitikz}[american currents]
\ctikzset{bipoles/length=1cm}

\draw 
(0, 0) node[op amp] (opamp) {}
(opamp.-) to [short , *-] (-0.6,0.35) -- (-3.2,0.35) to[I=$ $, invert, i_<=$I_{s}$] (-3.2,-2.05) node[ground]{}
(opamp.-) --(-0.85,1) to[R=$R_F$] (1,1) to [short, -*] (1,0) to [short, -o](2,0) node at(2.3,0){$V_{o}$}
(opamp.out) to (1,0) 
(opamp.+) -- (-0.85,-0.35) -- (-0.85,-1) node[ground]{};
\draw
(-1.5,0.35) to [short, o-*] (-2.2,0.35) to [R =$R_{s}$] (-2.2,-2.05) node[ground]{};
\draw
(-1.3,-1.9) -- (-1.3,0.2) to [short, i_>=$ $] (-0.8,0.2);
\draw
(-2.7,-1.9) -- (-2.7,0.2) to [short, i_>=$ $] (-2.5,0.2);
\draw
(1.3,-1.9) -- (1.3,-0.2) to [short, i_>=$ $] (0.8,-0.2);
\draw
node at (0,0){$\mu$}
node at (-1.3,-2.05){$R_{i n}$}
node at (-2.7,-1.9){$R_{i f}$}
node at (2.3,-2.05){$R_{o f}$}
node at (1.3,-2.05){$R_{o u t}$};
\draw
(2.3,-1.9) -- (2.3,-0.2) to [short, i_>=$ $] (1.8,-0.2);
\end{circuitikz}



}
	\end{center}
\caption{}
\label{fig:ee18btech11011_Norton_Representation}
\end{figure}
%
\begin{figure}[!ht]
	\begin{center}
			\resizebox{\columnwidth}{!}{\begin{circuitikz}[american voltages]
\draw
  (0,0) to [short, *-o] (-2,0) -- (-4,0) to [short, f_>=$I_{f}$] (-4,4) to [short, -o] (-2,4) to [R, l=$R_{F}$, -o] (2,4) -- (4,4) to [V, l_= $V_{o}$] (4,0) to [short, -o] (2,0) -- (0,0) -- (0,-1) node[ground]{};
  \end{circuitikz}



}
	\end{center}
\caption{}
\label{fig:ee18btech11011_beta_circuit}
\end{figure}
%
\begin{figure}[!ht]
	\begin{center}
			\resizebox{\columnwidth}{!}{\begin{circuitikz}
\draw
  (0,0) to [short, f>=$I_i$, o-*] (2,0) to [R = $R_{s}$] (2,-6) node[ground]{};
\draw
  (2,0) to [short, -*] (4,0) to [R = $R_{F}$] (4,-6) node[ground]{};
\draw
  (4,0) to [short, -*] (6,0) to [R = $R_{i d}$] (6,-6) node[ground]{};
\draw
  (6,0) to [short, -o] (8,0);
\draw
  (0.2,-5.4) -- (0.2,-0.6) to [short ,i_>=$ $] (1,-0.6);
\draw
  node at (8,-0.6){$+$}
  node at (8,-5.4){$-$}
  node at (8,-3){$V_{i d}$}
  node at (0.2,-5.8){$R_{i}$};
  \end{circuitikz}


}
	\end{center}
\caption{}
\label{fig:ee18btech11011_A1_circuit}
\end{figure}
%
\begin{figure}[!ht]
	\begin{center}
			\resizebox{\columnwidth}{!}{\begin{circuitikz}[american voltages]
\draw
(0,0) to [short, o-*] (-4,0) to [R = $R_{F}$] (-4,-6) node[ground]{} ;
\draw
(-4,0) to [R = $r_o$] (-6,0) to [cV = $-\mu V_{i d}$] (-6,-6) node[ground]{};
\draw
  (0.4,-5.4) -- (0.4,-0.6) to [short ,i_>=$ $] (-1,-0.6);
\draw
node at (0,-0.8){$+$}
node at (0,-5.2){$-$}
node at (0,-3){$V_{o}$}
node at (0.4,-5.7){$R_{o}$};

\end{circuitikz}


}
	\end{center}
\caption{}
\label{fig:ee18btech11011_A2_circuit}
\end{figure}

\item Write all the feedback equations based on all the Figs. using KCL/KVL.

\solution The equations are as follows:
%
\begin{align}
\label{eq:ee18btech11011_Closed_loop_Gain1}
T &= \frac{V_{o}}{I_{s}} = \frac{G}{1+GH}
\\
\label{eq:ee18btech11011_Feedback_Factor}
H &= \frac{I_{f}}{V_{o}} = -\frac{1}{R_F}
\\
\label{eq:ee18btech11011_I_i}
I_{i} &= I_{s} - I_{f} = \frac{I_{s}}{1+GH}
\\
\label{eq:ee18btech11011_R_if}
R_{i f} &= \frac{V_{i}}{I_{s}} = \frac{V_{i}}{(1+GH)I_{i}} =\frac{R_{i}}{1+GH}
\\
\label{eq:ee18btech11011_R_of}
R_{o f} &= \frac{R_{o}}{1+GH}
\\
\label{eq:ee18btech11011_R_in_and_R_out}
R_{i n} &= \frac{1}{\frac{1}{R_{i f}} - \frac{1}{R_{s}}} , R_{o u t} = \frac{1}{\frac{1}{R_{o f}} - \frac{1}{R_{L}}}
\end{align}
%

\begin{table}[!ht]
\centering
\input{./tables/ee18btech11011/Parameters_Table.tex}
\caption{}
\label{table: Parameters_Table}
\end{table}

\begin{figure}[!ht]
	\begin{center}
			\resizebox{\columnwidth}{!}{\tikzstyle{input} = [coordinate]
\tikzstyle{output} = [coordinate]
\tikzstyle{block} = [draw, rectangle]
\tikzstyle{sum} = [draw, circle]

\begin{tikzpicture}[auto, node distance=2cm,>=latex']
    \node [input, name=input] {$I_{s}$};
    \node [sum, right of=input] (sum) {};
    \node [block, right of=sum] (controller) {$G$};
    \node [output, right of=controller] (output) {};
    \node [block, below of=controller] (feedback) {$H$};
    \draw [draw,->] (input) -- node {$I_{s}$} (sum);
    \draw [->] (sum) -- node {$I_{i}$} (controller);
    \draw [->] (controller) -- node [name=y] {$I_{o}$}(output);
    \draw [->] (y) |- (feedback);
    \draw [->] (feedback) -| node[pos=0.95]{$-$}  node [near end] {$I_{f}$} (sum);
    \draw [->] (feedback) -| node[pos=1.15]{$+$}  node [near end] {} (sum);
\end{tikzpicture}
}
	\end{center}
\caption{Block Diagram}
\label{fig:Block Diagram}
\end{figure}

\item If the loop gain is very large, what approximate closed-loop voltage gain $V_{o}/V_{s}$ is realized?Also if $R_{s}$ = 1 k$\ohm$ , give the value of $R_{F}$ that will result in $V_{o}/V_{s}$ \simeq -10 V/V.

\solution If the loop gain GH is very large then the closed loop gain is,
%
\begin{align}
T = \frac{V_{o}}{I_{s}} = \frac{G}{1+GH}
\\
\because GH >> 1 \implies T \approx \frac{1}{H}
\label{eq:ee18btech11011_Closed_Loop_Gain2}
\end{align}
%
From equation \ref{eq:ee18btech11011_Feedback_Factor} and \ref{eq:ee18btech11011_Closed_Loop_Gain2} we get,
%
\begin{align}
T &\approx -R_{F}
\\
\implies \frac{V_{o}R_s}{V_{s}} &\approx -R_{F}
\\
\implies \frac{V_{o}}{V_{s}} &\approx \frac{-R_{F}}{R_{s}}
\\
\implies R_F &= 10k\ohm
\end{align}
%
\\
\item If the amplifier $\mu$ has a dc gain of $10^3$ V/V, an input resistance $R_{i d}$ = 100 k$\ohm$ , and an output resistance $r_{o}$ = 1 k$\ohm$ , find the actual $V_{o}/V_{s}$ realized. Also find $R_i_n$ and $R_{o u t}$.

\solution To find $V_{o}/V_{s]$, $R_{i n}$ and $R_{o u t}$ first find the other necessary parameters.

From Fig. \ref{fig:ee18btech11011_A1_circuit} we get,
%
\begin{align}
R_{i} &= R_{i d}\|R_{F}\|R_{s} 
\\
\label{ee18btech11011_R_i}
\imlies R_{i} &= 100k\|10k\|1k = \frac{1000}{1011}k = 0.90k\ohm
\\
\label{eq:ee18btech11011_V_id}
V_{i d} &= I_{i}R_{i}
\end{align}
%
From Fig. \ref{fig:ee18btech11011_A2_circuit} we get,
%
\begin{align}
R_{o} &= r_{o}\|R_{F}
\\
\label{ee18btech11011_R_o}
\implies R_{o} &= 1k\|10k = 0.91k\ohm
\\
\label{eq:ee18btech11011_V_o}
V_{o} &= -\mu V_{i d}\frac{R_{F}}{r_{o} + R_{F}}
\end{align}
%
From equation \ref{eq:ee18btech11011_V_id} and \ref{eq:ee18btech11011_V_o} we get the open-loop gain as,
%
\begin{align}
G &= \frac{V_{o}}{I_{i}} = -\mu R_{i}\frac{R_{F}}{r_{o} + R_{F}}
\\
\label{ee18btech11011_Open_loop_Gain}
\implies G &= -(1000)(0.90)\frac{10}{11} = -819.00k\ohm
\end{align}
%
From equation \ref{eq:ee18btech11011_Feedback_Factor} and \ref{ee18btech11011_Open_loop_Gain} we get closed loop gain T as,
%
\begin{align}
T &= \frac{G}{1+GH} = \frac{-819}{82.9} = -9.88k\ohm
\end{align}
%
From equation \ref{eq:ee18btech11011_Closed_loop_Gain1} we know,
\begin{align}
T &= \frac{V_{o}}{I_{s}}
\\
\implies T &= \frac{V_{o}R_{s}}{V_{s}}
\\
\implies \frac{V_{o}}{V_{s}} &= \frac{T}{R_{s}} 
\\
\implies \frac{V_{o}}{V_{s}} &= \frac{-9.88}{1} = -9.88V/V
\end{align}
From equation \ref{eq:ee18btech11011_R_if} and \ref{eq:ee18btech11011_R_in_and_R_out} we know,
%
\begin{align}
R_{i f} &= \frac{R_{i}}{1+GH} = \frac{0.90}{82.9}
\\
\implies R_{i f} &= 10.87\ohm
\\
R_{i n} &= \frac{1}{\frac{1}{R_{i f}} - \frac{1}{R_{s}}}
\\
\implies R_{i n} &= \frac{1}{\frac{1}{10.87} - \frac{1}{1000}} = 10.99\ohm
\end{align}
%
Because $R_{L}$ is not there in the circuit so we take it's value as $\infty$, so from equation \ref{eq:ee18btech11011_R_of} and \ref{eq:ee18btech11011_R_in_and_R_out} we know,
%
\begin{align}
R_{o f} &= \frac{R_{o}}{1+GH} = \frac{0.91}{82.9}
\\
\implies R_{o f} &= 10.97\ohm
\\
R_{o u t} &= \frac{1}{\frac{1}{R_{o f}} - \frac{1}{R_{L}}}
\\
\implies R_{o u t} &= \frac{1}{\frac{1}{10.97} - \frac{1}{\infty}} = 10.97\ohm
\end{align}
%
Verify the above calculations using the following Python code.
\begin{lstlisting}
codes/ee18btech11011/ee18btech11011_cal.ipynb
\end{lstlisting}
\end{enumerate}

