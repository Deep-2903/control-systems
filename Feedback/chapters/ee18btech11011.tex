\begin{enumerate}[label=\thesection.\arabic*.,ref=\thesection.\theenumi]
\numberwithin{equation}{enumi}

\item The circuit in Fig. \ref{fig:ee18btech11011_original_circuit} utilizes a voltage amplifier with gain $\mu$ in a shunt-shunt feedback topology with the feedback network composed of resistor $R_F$, to use the feedback equations convert the signal source to its Norton Representation.Also draw the H-Circuit and G-Circuit. 
\renewcommand{\thefigure}{\theenumi.\arabic{figure}}
%
\begin{figure}[!ht]
	\begin{center}
		
		\resizebox{\columnwidth}{!}{\input{./figs/ee18btech11011/original_circuit.tex}}
	\end{center}
\caption{}
\label{fig:ee18btech11011_original_circuit}
\end{figure}
%
\\
\solution  See Fig. \ref{fig:ee18btech11011_Norton_Representation} for the Norton Representation, Fig. \ref{fig:ee18btech11011_beta_circuit} for the H-Circuit , Fig.\ref{fig:ee18btech11011_A1_circuit} and Fig. \ref{fig:ee18btech11011_A2_circuit} for the G-Circuit. 
%
\begin{figure}[!ht]
	\begin{center}
			\resizebox{\columnwidth}{!}{\input{./figs/ee18btech11011/Shunt_Shunt_Amplifier_Block_Diagram.tex}}
	\end{center}
\caption{Shunt Shunt Amplifier Block Diagram}
\label{fig:Shunt_Shunt_Amplifier_Block_Diagram}
\end{figure}
%
\begin{figure}[!ht]
	\begin{center}
			\resizebox{\columnwidth}{!}{\input{./figs/ee18btech11011/Norton_Representation.tex}}
	\end{center}
\caption{}
\label{fig:ee18btech11011_Norton_Representation}
\end{figure}
%
\begin{figure}[!ht]
	\begin{center}
			\resizebox{\columnwidth}{!}{\input{./figs/ee18btech11011/beta_circuit.tex}}
	\end{center}
\caption{}
\label{fig:ee18btech11011_beta_circuit}
\end{figure}
%
\begin{figure}[!ht]
	\begin{center}
			\resizebox{\columnwidth}{!}{\input{./figs/ee18btech11011/A1_circuit.tex}}
	\end{center}
\caption{}
\label{fig:ee18btech11011_A1_circuit}
\end{figure}
%
\begin{figure}[!ht]
	\begin{center}
			\resizebox{\columnwidth}{!}{\input{./figs/ee18btech11011/A2_circuit.tex}}
	\end{center}
\caption{}
\label{fig:ee18btech11011_A2_circuit}
\end{figure}

\item Write all the feedback equations based on all the Figs. using KCL/KVL.

\solution The equations are as follows:
%
\begin{align}
\label{eq:ee18btech11011_Closed_loop_Gain1}
T &= \frac{V_{o}}{I_{s}} = \frac{G}{1+GH}
\\
\label{eq:ee18btech11011_Feedback_Factor}
H &= \frac{I_{f}}{V_{o}} = -\frac{1}{R_F}
\\
\label{eq:ee18btech11011_I_i}
I_{i} &= I_{s} - I_{f} = \frac{I_{s}}{1+GH}
\\
\label{eq:ee18btech11011_R_if}
R_{i f} &= \frac{V_{i}}{I_{s}} = \frac{V_{i}}{(1+GH)I_{i}} =\frac{R_{i}}{1+GH}
\\
\label{eq:ee18btech11011_R_of}
R_{o f} &= \frac{R_{o}}{1+GH}
\\
\label{eq:ee18btech11011_R_in_and_R_out}
R_{i n} &= \frac{1}{\frac{1}{R_{i f}} - \frac{1}{R_{s}}} , R_{o u t} = \frac{1}{\frac{1}{R_{o f}} - \frac{1}{R_{L}}}
\end{align}
%

\begin{table}[!ht]
\centering
\input{./tables/ee18btech11011/Parameters_Table.tex}
\caption{}
\label{table: Parameters_Table}
\end{table}

\begin{figure}[!ht]
	\begin{center}
			\resizebox{\columnwidth}{!}{\input{./figs/ee18btech11011/Block_Diagram.tex}}
	\end{center}
\caption{Block Diagram}
\label{fig:Block Diagram}
\end{figure}

\item If the loop gain is very large, what approximate closed-loop voltage gain $V_o$/$V_s$ is realized?Also if $R_s$ = 1 k$\ohm$ , give the value of $R_F$ that will result in $V_o$/$V_s$ \simeq -10 V/V.

\solution If the loop gain GH is very large then the closed loop gain is,
%
\begin{align}
T = \frac{V_{o}}{I_{s}} = \frac{G}{1+GH}
\\
\because GH >> 1 \implies T \approx \frac{1}{H}
\label{eq:ee18btech11011_Closed_Loop_Gain2}
\end{align}
%
From equation \ref{eq:ee18btech11011_Feedback_Factor} and \ref{eq:ee18btech11011_Closed_Loop_Gain2} we get,
%
\begin{align}
T &\approx -R_{F}
\\
\implies \frac{V_{o}R_s}{V_{s}} &\approx -R_{F}
\\
\implies \frac{V_{o}}{V_{s}} &\approx \frac{-R_{F}}{R_{s}}
\\
\implies R_F &= 10k\ohm
\end{align}
%
\\
\item If the amplifier $\mu$ has a dc gain of $10^3$ V/V, an input resistance $R_{i d}$ = 100 k$\ohm$ , and an output resistance $r_o$ = 1 k$\ohm$ , find the actual $V_o$/$V_s$ realized. Also find $R_i_n$ and $R_o_u_t$.

\solution To find $V_o$/$V_s$, $R_i_n$ and $R_o_u_t$ first find the other necessary parameters.

From Fig. \ref{fig:ee18btech11011_A1_circuit} we get,
%
\begin{align}
R_{i} &= R_{i d}\|R_{F}\|R_{s} 
\\
\label{ee18btech11011_R_i}
\imlies R_{i} &= 100k\|10k\|1k = \frac{1000}{1011}k = 0.90k\ohm
\\
\label{eq:ee18btech11011_V_id}
V_{i d} &= I_{i}R_{i}
\end{align}
%
From Fig. \ref{fig:ee18btech11011_A2_circuit} we get,
%
\begin{align}
R_{o} &= r_{o}\|R_{F}
\\
\label{ee18btech11011_R_o}
\implies R_{o} &= 1k\|10k = 0.91k\ohm
\\
\label{eq:ee18btech11011_V_o}
V_{o} &= -\mu V_{i d}\frac{R_{F}}{r_{o} + R_{F}}
\end{align}
%
From equation \ref{eq:ee18btech11011_V_id} and \ref{eq:ee18btech11011_V_o} we get the open-loop gain as,
%
\begin{align}
G &= \frac{V_{o}}{I_{i}} = -\mu R_{i}\frac{R_{F}}{r_{o} + R_{F}}
\\
\label{ee18btech11011_Open_loop_Gain}
\implies G &= -(1000)(0.90)\frac{10}{11} = -819.00k\ohm
\end{align}
%
From equation \ref{eq:ee18btech11011_Feedback_Factor} and \ref{ee18btech11011_Open_loop_Gain} we get closed loop gain T as,
%
\begin{align}
T &= \frac{G}{1+GH} = \frac{-819}{82.9} = -9.88k\ohm
\end{align}
%
From equation \ref{eq:ee18btech11011_Closed_loop_Gain1} we know,
\begin{align}
T &= \frac{V_{o}}{I_{s}}
\\
\implies T &= \frac{V_{o}R_{s}}{V_{s}}
\\
\implies \frac{V_{o}}{V_{s}} &= \frac{T}{R_{s}} 
\\
\implies \frac{V_{o}}{V_{s}} &= \frac{-9.88}{1} = -9.88V/V
\end{align}
From equation \ref{eq:ee18btech11011_R_if} and \ref{eq:ee18btech11011_R_in_and_R_out} we know,
%
\begin{align}
R_{i f} &= \frac{R_{i}}{1+GH} = \frac{0.90}{82.9}
\\
\implies R_{i f} &= 10.87\ohm
\\
R_{i n} &= \frac{1}{\frac{1}{R_{i f}} - \frac{1}{R_{s}}}
\\
\implies R_{i n} &= \frac{1}{\frac{1}{10.87} - \frac{1}{1000}} = 10.99\ohm
\end{align}
%
Because $R_L$ is not there in the circuit so we take it's value as $\infty$, so from equation \ref{eq:ee18btech11011_R_of} and \ref{eq:ee18btech11011_R_in_and_R_out} we know,
%
\begin{align}
R_{o f} &= \frac{R_{o}}{1+GH} = \frac{0.91}{82.9}
\\
\implies R_{o f} &= 10.97\ohm
\\
R_{o u t} &= \frac{1}{\frac{1}{R_{o f}} - \frac{1}{R_{L}}}
\\
\implies R_{o u t} &= \frac{1}{\frac{1}{10.97} - \frac{1}{\infty}} = 10.97\ohm
\end{align}
%
Verify the above calculations using the following Python code.
\begin{lstlisting}
codes/ee18btech11011/ee18btech11011_cal.ipynb
\end{lstlisting}
\end{enumerate}

