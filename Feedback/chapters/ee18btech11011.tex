The circuit in Fig. \ref{fig:ee18btech11011_original_circuit} utilizes a voltage amplifier with gain $\mu$ in a shunt-shunt feedback topology with the feedback network composed of resistor $R_{F}$.In order to be able to use the feedback equations you should first convert the signal source to it's Norton Representation.
\begin{enumerate}
    \item If the loop gain is very large, what approximate closed loop voltage gain $V_{o}/V_{s}$ is realized? If $R_{s}$ = 1k$\ohm$, give the value of $R_{F}$ that will result in $V_{o}/V_{s}$ $\simeq$ -10V/V.
    \item If the amplifier $\mu$ has a dc gain of $10^{3}$ V/V, an input
    resistance $R_{i d}$ = 100k$\ohm$, and an output resistance $r_{o}$ = 1k$\ohm$
    , find the actual $V_{o}/V_{s}$ realized. Also find $R_{i n}$ and $R_{o u t}$.
    \item If the amplifier $\mu$ has an upper 3-dB frequency of 1 kHz and a uniform -20-dB/decade gain rolloff, what is the 3-dB frequency of the gain $\mid V_{o}/V_{s} \mid$.
\end{enumerate}

\begin{enumerate}[label=\arabic*.,ref=\theenumi]
\numberwithin{equation}{enumi}
\item Fig. \ref{fig:ee18btech11011_original_circuit} shows the original circuit. Draw the Norton Representation, A-Circuit and H-Circuit.
\\
\solution See Fig. \ref{fig:ee18btech11011_Norton_Representation} for the Norton Representation, Fig. \ref{fig:ee18btech11011_A1_circuit} and Fig. \ref{fig:ee18btech11011_A2_circuit} for A-Circuit and Fig. \ref{fig:ee18btech11011_beta_circuit} for the H-Circuit.  

\renewcommand{\thefigure}{\theenumi.\arabic{figure}}
\begin{figure}[!ht]
	\begin{center}
		\resizebox{\columnwidth}{!}{\begin{circuitikz}[american voltages]
\ctikzset{bipoles/length=1cm}

\draw 
(0, 0) node[op amp] (opamp) {}
(opamp.-) to [short , *-] (-0.6,0.35) to[R =$R_{s}$] (-2.6,0.35) to[V=$V_{s}$] (-2.6,-2.05) node[ground]{}
(opamp.-) --(-0.85,1) to[R=$R_{F}$] (1,1) to [short, -*] (1,0) to [short, -o](2,0) node at(2.3,0){$V_{o}$}
(opamp.out) to (1,0) 
(opamp.+) -- (-0.85,-0.35) -- (-0.85,-1) node[ground]{};
\draw
(-1.3,-1.9) -- (-1.3,-0.2) to [short, i_>=$ $] (-0.8,-0.2);
\draw
node at (-1.3,-2.05){$R_{i n}$}
node at (0,0){$\mu$}
node at (2.3,-2.05){$R_{o u t}$};
\draw
(2.3,-1.9) -- (2.3,-0.2) to [short, i_>=$ $] (1.5,-0.2);
\end{circuitikz}

}
	\end{center}
\caption{}
\label{fig:ee18btech11011_original_circuit}
\end{figure}
%
\begin{figure}[!ht]
	\begin{center}
			\resizebox{\columnwidth}{!}{\begin{circuitikz}[american]
\usetikzlibrary{positioning, fit, calc}
\draw (0,0) to[I = $I_{s}$] (0,2) -- (2,2) to[R=$R_{s}$,*-*] (2,0){}
(2,2) to [short, -o] (2.5,2) -- (5,2) {}
(7,1)node[draw,minimum width=4cm,minimum height=4cm] (load) {Gain Amplifier}{}
(7,-4)node[draw,minimum width=4cm,minimum height=4cm] (load) {Feedback Network}{}
(0,0) to [short, -o] (2.5,0) -- (5,0)
(13,0) to [R=$R_{L}$,*-*] (13,2) to [short, -o] (12,2) to[short, i = $I_{o}$] (9,2)
(3,0) to [short, *-] (3,-5) -- (5,-5){}
(14,2) to [short, o-] (13,2){}
(4,2) to [short, *-] (4,-3) -- (5,-3){}
(10,2) to [short, *-] (10,-3) -- (9,-3){}
(9,0) to [short, -*] (11,0) to [short, -o] (12,0){}
(9,-5) -- (11,-5) -- (11,0) to [short, -o] (14,0){}
;
\draw
node at (14,0.2){$-$}
node at (14,1.8){$+$}
node at (14.3,1){$V_{o}$};
\end{circuitikz}
}
	\end{center}
\caption{Shunt Shunt Amplifier Block Diagram}
\label{fig:Shunt_Shunt_Amplifier_Block_Diagram}
\end{figure}
%
\begin{figure}[!ht]
	\begin{center}
			\resizebox{\columnwidth}{!}{\begin{circuitikz}[american currents]
\ctikzset{bipoles/length=1cm}

\draw 
(0, 0) node[op amp] (opamp) {}
(opamp.-) to [short , *-] (-0.6,0.35) -- (-3.2,0.35) to[I=$ $, invert, i_<=$I_{s}$] (-3.2,-2.05) node[ground]{}
(opamp.-) --(-0.85,1) to[R=$R_F$] (1,1) to [short, -*] (1,0) to [short, -o](2,0) node at(2.3,0){$V_{o}$}
(opamp.out) to (1,0) 
(opamp.+) -- (-0.85,-0.35) -- (-0.85,-1) node[ground]{};
\draw
(-1.5,0.35) to [short, o-*] (-2.2,0.35) to [R =$R_{s}$] (-2.2,-2.05) node[ground]{};
\draw
(-1.3,-1.9) -- (-1.3,0.2) to [short, i_>=$ $] (-0.8,0.2);
\draw
(-2.7,-1.9) -- (-2.7,0.2) to [short, i_>=$ $] (-2.5,0.2);
\draw
(1.3,-1.9) -- (1.3,-0.2) to [short, i_>=$ $] (0.8,-0.2);
\draw
node at (0,0){$\mu$}
node at (-1.3,-2.05){$R_{i n}$}
node at (-2.7,-1.9){$R_{i f}$}
node at (2.3,-2.05){$R_{o f}$}
node at (1.3,-2.05){$R_{o u t}$};
\draw
(2.3,-1.9) -- (2.3,-0.2) to [short, i_>=$ $] (1.8,-0.2);
\end{circuitikz}



}
	\end{center}
\caption{}
\label{fig:ee18btech11011_Norton_Representation}
\end{figure}
%
\begin{figure}[!ht]
	\begin{center}
			\resizebox{\columnwidth}{!}{\begin{circuitikz}
\draw
  (0,0) to [short, f>=$I_i$, o-*] (2,0) to [R = $R_{s}$] (2,-6) node[ground]{};
\draw
  (2,0) to [short, -*] (4,0) to [R = $R_{F}$] (4,-6) node[ground]{};
\draw
  (4,0) to [short, -*] (6,0) to [R = $R_{i d}$] (6,-6) node[ground]{};
\draw
  (6,0) to [short, -o] (8,0);
\draw
  (0.2,-5.4) -- (0.2,-0.6) to [short ,i_>=$ $] (1,-0.6);
\draw
  node at (8,-0.6){$+$}
  node at (8,-5.4){$-$}
  node at (8,-3){$V_{i d}$}
  node at (0.2,-5.8){$R_{i}$};
  \end{circuitikz}


}
	\end{center}
\caption{}
\label{fig:ee18btech11011_A1_circuit}
\end{figure}
%
\begin{figure}[!ht]
	\begin{center}
			\resizebox{\columnwidth}{!}{\begin{circuitikz}[american voltages]
\draw
(0,0) to [short, o-*] (-4,0) to [R = $R_{F}$] (-4,-6) node[ground]{} ;
\draw
(-4,0) to [R = $r_o$] (-6,0) to [cV = $-\mu V_{i d}$] (-6,-6) node[ground]{};
\draw
  (0.4,-5.4) -- (0.4,-0.6) to [short ,i_>=$ $] (-1,-0.6);
\draw
node at (0,-0.8){$+$}
node at (0,-5.2){$-$}
node at (0,-3){$V_{o}$}
node at (0.4,-5.7){$R_{o}$};

\end{circuitikz}


}
	\end{center}
\caption{}
\label{fig:ee18btech11011_A2_circuit}
\end{figure}
%
\begin{figure}[!ht]
	\begin{center}
			\resizebox{\columnwidth}{!}{\begin{circuitikz}[american voltages]
\draw
  (0,0) to [short, *-o] (-2,0) -- (-4,0) to [short, f_>=$I_{f}$] (-4,4) to [short, -o] (-2,4) to [R, l=$R_{F}$, -o] (2,4) -- (4,4) to [V, l_= $V_{o}$] (4,0) to [short, -o] (2,0) -- (0,0) -- (0,-1) node[ground]{};
  \end{circuitikz}



}
	\end{center}
\caption{}
\label{fig:ee18btech11011_beta_circuit}
\end{figure}
%
\begin{figure}[!ht]
	\begin{center}
			\resizebox{\columnwidth}{!}{\tikzstyle{input} = [coordinate]
\tikzstyle{output} = [coordinate]
\tikzstyle{block} = [draw, rectangle]
\tikzstyle{sum} = [draw, circle]

\begin{tikzpicture}[auto, node distance=2cm,>=latex']
    \node [input, name=input] {$I_{s}$};
    \node [sum, right of=input] (sum) {};
    \node [block, right of=sum] (controller) {$G$};
    \node [output, right of=controller] (output) {};
    \node [block, below of=controller] (feedback) {$H$};
    \draw [draw,->] (input) -- node {$I_{s}$} (sum);
    \draw [->] (sum) -- node {$I_{i}$} (controller);
    \draw [->] (controller) -- node [name=y] {$I_{o}$}(output);
    \draw [->] (y) |- (feedback);
    \draw [->] (feedback) -| node[pos=0.95]{$-$}  node [near end] {$I_{f}$} (sum);
    \draw [->] (feedback) -| node[pos=1.15]{$+$}  node [near end] {} (sum);
\end{tikzpicture}
}
	\end{center}
\caption{Block Diagram}
\label{fig:Block Diagram}
\end{figure}
%
\item Refer Table \ref{table:ee18btech11011_Parameters_table} for the parameters.
%
\begin{table}[!ht]
\centering
\input{./tables/ee18btech11011/Parameters_Table.tex}
\caption{}
\label{table:ee18btech11011_Parameters_table}
\end{table}
%
\item Write all the feedback equations based on all the Figs. using KCL/KVL.
\\
\solution The equations are as follows:
%
\begin{align}
\label{eq:ee18btech11011_Closed_loop_Gain1}
T &= \frac{V_{o}}{I_{s}} = \frac{G}{1+GH}\\
\label{eq:ee18btech11011_Feedback_Factor}
H &= \frac{I_{f}}{V_{o}} = -\frac{1}{R_F}\\
\label{eq:ee18btech11011_I_i}
I_{i} &= I_{s} - I_{f} = \frac{I_{s}}{1+GH}\\
\label{eq:ee18btech11011_R_if}
R_{i f} &= \frac{V_{i}}{I_{s}} = \frac{V_{i}}{(1+GH)I_{i}} =\frac{R_{i}}{1+GH}\\
\label{eq:ee18btech11011_R_of}
R_{o f} &= \frac{R_{o}}{1+GH}\\
\label{eq:ee18btech11011_R_in_and_R_out}
R_{i n} &= \frac{1}{\frac{1}{R_{i f}} - \frac{1}{R_{s}}} , R_{o u t} = \frac{1}{\frac{1}{R_{o f}} - \frac{1}{R_{L}}}
\end{align}
%
\item If the loop gain is very large, what approximate closed-loop voltage gain $V_{o}/V_{s}$ is realized?Also if $R_{s}$ = 1 k$\ohm$ , give the value of $R_{F}$ that will result in $V_{o}/V_{s}$ $\simeq$ -10 V/V.
\\
\solution If the loop gain GH is very large then the closed loop gain is,

\begin{align}
T = \frac{V_{o}}{I_{s}} = \frac{G}{1+GH}\\
\because GH >> 1 \implies T \approx \frac{1}{H}
\label{eq:ee18btech11011_Closed_Loop_Gain2}
\end{align}

From equation \ref{eq:ee18btech11011_Feedback_Factor} and \ref{eq:ee18btech11011_Closed_Loop_Gain2} we get,

\begin{align}
T &\approx -R_{F}\\
\implies \frac{V_{o}R_s}{V_{s}} &\approx -R_{F}\\
\implies R_{F} &= 10k\ohm
\end{align}
\item If the amplifier $\mu$ has a dc gain of $10^3$ V/V, an input resistance $R_{i d}$ = 100 k$\ohm$ , and an output resistance $r_{o}$ = 1 k$\ohm$ , find the actual $V_{o}/V_{s}$ realized. Also find $R_{i n}$ and $R_{o u t}$.
\\
\solution To find $V_{o}/V_{s}$, $R_{i n}$ and $R_{o u t}$ first find the other necessary parameters.
From Fig. \ref{fig:ee18btech11011_A1_circuit} we get,

\begin{align}
R_{i} &= R_{i d}\|R_{F}\|R_{s}\\
\label{ee18btech11011_R_i}
R_{i} &= 100k\|10k\|1k = 0.90k\ohm\\
\label{eq:ee18btech11011_V_id}
V_{i d} &= I_{i}R_{i}
\end{align}

From Fig. \ref{fig:ee18btech11011_A2_circuit} we get,

\begin{align}
R_{o} &= r_{o}\|R_{F}\\
\label{ee18btech11011_R_o}
\implies R_{o} &= 1k\|10k = 0.91k\ohm\\
\label{eq:ee18btech11011_V_o}
V_{o} &= -\mu V_{i d}\frac{R_{F}}{r_{o} + R_{F}}
\end{align}

From equation \ref{eq:ee18btech11011_V_id} and \ref{eq:ee18btech11011_V_o} we get the open-loop gain as,

\begin{align}
G &= \frac{V_{o}}{I_{i}} = -\mu R_{i}\frac{R_{F}}{r_{o} + R_{F}}
\\
\label{ee18btech11011_Open_loop_Gain}
\implies G &= -(1000)(0.90)\frac{10}{11} = -819.00k\ohm
\end{align}

From equation \ref{eq:ee18btech11011_Feedback_Factor} and \ref{ee18btech11011_Open_loop_Gain} we get closed loop gain T as,

\begin{align}
T &= \frac{G}{1+GH} = \frac{-819}{82.9} = -9.88k\ohm
\end{align}

From equation \ref{eq:ee18btech11011_Closed_loop_Gain1} we know,
\begin{align}
T &= \frac{V_{o}}{I_{s}}
\\
\implies T &= \frac{V_{o}R_{s}}{V_{s}}
\\
\implies \frac{V_{o}}{V_{s}} &= \frac{T}{R_{s}} 
\\
\implies \frac{V_{o}}{V_{s}} &= \frac{-9.88}{1} = -9.88V/V
\label{eq:ee18btech11011_Value_of_Vo/Vs}
\end{align}
From equation \ref{eq:ee18btech11011_R_if} and \ref{eq:ee18btech11011_R_in_and_R_out} we know,

\begin{align}
R_{i f} &= \frac{R_{i}}{1+GH} = \frac{0.90}{82.9}
\\
\implies R_{i f} &= 10.87\ohm
\\
R_{i n} &= \frac{1}{\frac{1}{R_{i f}} - \frac{1}{R_{s}}}
\\
\implies R_{i n} &= \frac{1}{\frac{1}{10.87} - \frac{1}{1000}} = 10.99\ohm
\end{align}

Because $R_{L}$ is not there in the circuit so we take it's value as $\infty$, so from equation \ref{eq:ee18btech11011_R_of} and \ref{eq:ee18btech11011_R_in_and_R_out} we know,

\begin{align}
R_{o f} &= \frac{R_{o}}{1+GH} = \frac{0.91}{82.9}
\\
\implies R_{o f} &= 10.97\ohm
\\
R_{o u t} &= \frac{1}{\frac{1}{R_{o f}} - \frac{1}{R_{L}}}
\\
\implies R_{o u t} &= \frac{1}{\frac{1}{10.97} - \frac{1}{\infty}} = 10.97\ohm
\end{align}

Verify the above calculations using the following Python code.
\begin{lstlisting}
codes/ee18btech11011/ee18btech11011_cal.ipynb
\end{lstlisting}

\item If the amplifier $\mu$ has an upper 3-dB frequency of 1 kHz and a uniform -20-dB/decade gain rolloff, what is the 3-dB frequency of the gain $\mid V_{o}/V_{s}\mid$.
\\
\solution To find the 3-dB frequency i.e., $\omega_{3 d B}$ we need to look at the Fig.\ref{fig:ee18btech11011_Inverting_configuration}.
\begin{figure}[!ht]
	\begin{center}
			\resizebox{\columnwidth}{!}{\begin{circuitikz}[american voltages]
\ctikzset{bipoles/length=1cm}

\draw 
(0, 0) node[op amp] (opamp) {}
(opamp.-) to [short , *-] (-0.6,0.35) to[R =$R_{s}$] (-2.6,0.35) to[V=$V_{s}$, i<_=$I_{s}$] (-2.6,-2.05) node[ground]{}
(opamp.-) --(-0.85,1) to[R=$R_{F}$] (1,1) to [short, i>_=$I_{s}$, -*] (1,0) to [short, -o](2,0) node at(2.3,0){$V_{o}$}
(opamp.out) to (1,0) 
(opamp.+) -- (-0.85,-0.35) -- (-0.85,-1) node[ground]{};
\draw
node at (0,0){$\mu$}
node at (-0.85,0.07){$V_{i n}$}
node at (-2.2,0.7){$+$}
node at (-1.1,0.7){$-$}
node at (-1.55,0.7){$V_{f}$};
\end{circuitikz}

}
	\end{center}
\caption{}
\label{fig:ee18btech11011_Inverting_configuration}
\end{figure}

The open loop gain G is given as follows in terms of frequency:
\begin{align}
    G &= \frac{\mu}{1 + \frac{jf}{f_{c}}}
\end{align}

From Fig.\ref{fig:ee18btech11011_Inverting_configuration} we can say that:
\begin{align}
    V_{i n} &= V_{s} - V_{f}
    \label{eq:ee18btech11011_Vin_Vs_Vf_relation}
    \\
    V_{o} &= -GV_{i n}
    \label{eq:ee18btech11011_Vin_Vo_relation}
    \\
    \frac{V_{f}}{R_{s}} &= \frac{V_{i n} - V_{o}}{R_{F}}
    \label{eq:ee18btech11011_Vf_Vin_Vo_relation}
\end{align}

From equation \ref{eq:ee18btech11011_Vin_Vo_relation} and \ref{eq:ee18btech11011_Vf_Vin_Vo_relation} we get:
\begin{align}
    \frac{V_{f}}{R_{s}} &= \frac{-\frac{V_{o}}{G} - V_{o}}{R_{F}}
    \\
    \implies \frac{V_{f}}{V_{o}} &= -\frac{(1 + G)}{G}\frac{(R_{s})}{(R_{F})} = -H
    \label{eq:ee18btech11011_Transfer_Function}
    \\
    \because G>>1 \implies H &= \frac{R_{s}}{R_{F}}
\end{align}

Now from equation \ref{eq:ee18btech11011_Vin_Vs_Vf_relation}, \ref{eq:ee18btech11011_Vin_Vo_relation} and \ref{eq:ee18btech11011_Transfer_Function} we get:
\begin{align}
    -\frac{V_{o}}{G} &= V_{s} + HV_{o}
    \\
    \implies \frac{V_{o}}{V_{s}} &= -\frac{G}{1 + GH}
\end{align}

Now, for "f" to be 3-dB frequency given condition should be match i.e.,:
\begin{align}
    \mid\frac{V_{o}}{V_{s}}\mid &= \frac{1}{\sqrt{2}}
    \\
    \implies \mid-\frac{G}{1 + GH}\mid &= \frac{1}{\sqrt{2}}
    \\
    \implies \frac{\frac{\mu}{1 + \frac{jf}{f_{c}}}}{1 + \frac{(R_{s})}{(R_{F})}\frac{\mu}{1 + \frac{jf}{f_{c}}}} &= \frac{1}{\sqrt{2}}
\end{align}
 
\begin{table}[!ht]
\centering
%%%%%%%%%%%%%%%%%%%%%%%%%%%%%%%%%%%%%%%%%%%%%%%%%%%%%%%%%%%%%%%%%%%%%%
%%                                                                  %%
%%  This is the header of a LaTeX2e file exported from Gnumeric.    %%
%%                                                                  %%
%%  This file can be compiled as it stands or included in another   %%
%%  LaTeX document. The table is based on the longtable package so  %%
%%  the longtable options (headers, footers...) can be set in the   %%
%%  preamble section below (see PRAMBLE).                           %%
%%                                                                  %%
%%  To include the file in another, the following two lines must be %%
%%  in the including file:                                          %%
%%        \def\inputGnumericTable{}                                 %%
%%  at the beginning of the file and:                               %%
%%        \input{name-of-this-file.tex}                             %%
%%  where the table is to be placed. Note also that the including   %%
%%  file must use the following packages for the table to be        %%
%%  rendered correctly:                                             %%
%%    \usepackage[latin1]{inputenc}                                 %%
%%    \usepackage{color}                                            %%
%%    \usepackage{array}                                            %%
%%    \usepackage{longtable}                                        %%
%%    \usepackage{calc}                                             %%
%%    \usepackage{multirow}                                         %%
%%    \usepackage{hhline}                                           %%
%%    \usepackage{ifthen}                                           %%
%%  optionally (for landscape tables embedded in another document): %%
%%    \usepackage{lscape}                                           %%
%%                                                                  %%
%%%%%%%%%%%%%%%%%%%%%%%%%%%%%%%%%%%%%%%%%%%%%%%%%%%%%%%%%%%%%%%%%%%%%%



%%  This section checks if we are begin input into another file or  %%
%%  the file will be compiled alone. First use a macro taken from   %%
%%  the TeXbook ex 7.7 (suggestion of Han-Wen Nienhuys).            %%
\def\ifundefined#1{\expandafter\ifx\csname#1\endcsname\relax}


%%  Check for the \def token for inputed files. If it is not        %%
%%  defined, the file will be processed as a standalone and the     %%
%%  preamble will be used.                                          %%
\ifundefined{inputGnumericTable}

%%  We must be able to close or not the document at the end.        %%
	\def\gnumericTableEnd{\end{document}}


%%%%%%%%%%%%%%%%%%%%%%%%%%%%%%%%%%%%%%%%%%%%%%%%%%%%%%%%%%%%%%%%%%%%%%
%%                                                                  %%
%%  This is the PREAMBLE. Change these values to get the right      %%
%%  paper size and other niceties.                                  %%
%%                                                                  %%
%%%%%%%%%%%%%%%%%%%%%%%%%%%%%%%%%%%%%%%%%%%%%%%%%%%%%%%%%%%%%%%%%%%%%%

	\documentclass[12pt%
			  %,landscape%
                    ]{report}
       \usepackage[latin1]{inputenc}
       \usepackage{fullpage}
       \usepackage{color}
       \usepackage{array}
       \usepackage{longtable}
       \usepackage{calc}
       \usepackage{multirow}
       \usepackage{hhline}
       \usepackage{ifthen}



%%  End of the preamble for the standalone. The next section is for %%
%%  documents which are included into other LaTeX2e files.          %%
\else

%%  We are not a stand alone document. For a regular table, we will %%
%%  have no preamble and only define the closing to mean nothing.   %%
    \def\gnumericTableEnd{}

%%  If we want landscape mode in an embedded document, comment out  %%
%%  the line above and uncomment the two below. The table will      %%
%%  begin on a new page and run in landscape mode.                  %%
%       \def\gnumericTableEnd{\end{landscape}}
%       \begin{landscape}


%%  End of the else clause for this file being \input.              %%
\fi

%%%%%%%%%%%%%%%%%%%%%%%%%%%%%%%%%%%%%%%%%%%%%%%%%%%%%%%%%%%%%%%%%%%%%%
%%                                                                  %%
%%  The rest is the gnumeric table, except for the closing          %%
%%  statement. Changes below will alter the table's appearance.     %%
%%                                                                  %%
%%%%%%%%%%%%%%%%%%%%%%%%%%%%%%%%%%%%%%%%%%%%%%%%%%%%%%%%%%%%%%%%%%%%%%

\providecommand{\gnumericmathit}[1]{#1} 
%%  Uncomment the next line if you would like your numbers to be in %%
%%  italics if they are italizised in the gnumeric table.           %%
%\renewcommand{\gnumericmathit}[1]{\mathit{#1}}
\providecommand{\gnumericPB}[1]%
{\let\gnumericTemp=\\#1\let\\=\gnumericTemp\hspace{0pt}}
 \ifundefined{gnumericTableWidthDefined}
        \newlength{\gnumericTableWidth}
        \newlength{\gnumericTableWidthComplete}
        \newlength{\gnumericMultiRowLength}
        \global\def\gnumericTableWidthDefined{}
 \fi
%% The following setting protects this code from babel shorthands.  %%
 \ifthenelse{\isundefined{\languageshorthands}}{}{\languageshorthands{english}}
%%  The default table format retains the relative column widths of  %%
%%  gnumeric. They can easily be changed to c, r or l. In that case %%
%%  you may want to comment out the next line and uncomment the one %%
%%  thereafter                                                      %%
\providecommand\gnumbox{\makebox[0pt]}
%%\providecommand\gnumbox[1][]{\makebox}

%% to adjust positions in multirow situations                       %%
\setlength{\bigstrutjot}{\jot}
\setlength{\extrarowheight}{\doublerulesep}

%%  The \setlongtables command keeps column widths the same across  %%
%%  pages. Simply comment out next line for varying column widths.  %%
\setlongtables

\setlength\gnumericTableWidth{%
	53pt+%
	163pt+%
0pt}
\def\gumericNumCols{2}
\setlength\gnumericTableWidthComplete{\gnumericTableWidth+%
         \tabcolsep*\gumericNumCols*2+\arrayrulewidth*\gumericNumCols}
\ifthenelse{\lengthtest{\gnumericTableWidthComplete > \linewidth}}%
         {\def\gnumericScale{\ratio{\linewidth-%
                        \tabcolsep*\gumericNumCols*2-%
                        \arrayrulewidth*\gumericNumCols}%
{\gnumericTableWidth}}}%
{\def\gnumericScale{1}}

%%%%%%%%%%%%%%%%%%%%%%%%%%%%%%%%%%%%%%%%%%%%%%%%%%%%%%%%%%%%%%%%%%%%%%
%%                                                                  %%
%% The following are the widths of the various columns. We are      %%
%% defining them here because then they are easier to change.       %%
%% Depending on the cell formats we may use them more than once.    %%
%%                                                                  %%
%%%%%%%%%%%%%%%%%%%%%%%%%%%%%%%%%%%%%%%%%%%%%%%%%%%%%%%%%%%%%%%%%%%%%%

\ifthenelse{\isundefined{\gnumericColA}}{\newlength{\gnumericColA}}{}\settowidth{\gnumericColA}{\begin{tabular}{@{}p{53pt*\gnumericScale}@{}}x\end{tabular}}
\ifthenelse{\isundefined{\gnumericColB}}{\newlength{\gnumericColB}}{}\settowidth{\gnumericColB}{\begin{tabular}{@{}p{163pt*\gnumericScale}@{}}x\end{tabular}}

\begin{tabular}[c]{%
	b{\gnumericColA}%
	b{\gnumericColB}%
	}

%%%%%%%%%%%%%%%%%%%%%%%%%%%%%%%%%%%%%%%%%%%%%%%%%%%%%%%%%%%%%%%%%%%%%%
%%  The longtable options. (Caption, headers... see Goosens, p.124) %%
%	\caption{The Table Caption.}             \\	%
% \hline	% Across the top of the table.
%%  The rest of these options are table rows which are placed on    %%
%%  the first, last or every page. Use \multicolumn if you want.    %%

%%  Header for the first page.                                      %%
%	\multicolumn{2}{c}{The First Header} \\ \hline 
%	\multicolumn{1}{c}{colTag}	%Column 1
%	&\multicolumn{1}{c}{colTag}	\\ \hline %Last column
%	\endfirsthead

%%  The running header definition.                                  %%
%	\hline
%	\multicolumn{2}{l}{\ldots\small\slshape continued} \\ \hline
%	\multicolumn{1}{c}{colTag}	%Column 1
%	&\multicolumn{1}{c}{colTag}	\\ \hline %Last column
%	\endhead

%%  The running footer definition.                                  %%
%	\hline
%	\multicolumn{2}{r}{\small\slshape continued\ldots} \\
%	\endfoot

%%  The ending footer definition.                                   %%
%	\multicolumn{2}{c}{That's all folks} \\ \hline 
%	\endlastfoot
%%%%%%%%%%%%%%%%%%%%%%%%%%%%%%%%%%%%%%%%%%%%%%%%%%%%%%%%%%%%%%%%%%%%%%

\hhline{|-|-}
	 \multicolumn{1}{|p{\gnumericColA}|}%
	{\gnumericPB{\centering}\gnumbox{\textbf{Parameters}}}
	&\multicolumn{1}{p{\gnumericColB}|}%
	{\gnumericPB{\centering}\gnumbox{\textbf{Values}}}
\\
\hhline{|--|}
	 \multicolumn{1}{|p{\gnumericColA}|}%
	{\gnumericPB{\raggedright}\gnumbox[l]{$R_{s}$}}
	&\multicolumn{1}{p{\gnumericColB}|}%
	{\gnumericPB{\raggedright}\gnumbox[l]{1k$\ohm$}}
\\
\hhline{|--|}
	 \multicolumn{1}{|p{\gnumericColA}|}%
	{\gnumericPB{\raggedright}\gnumbox[l]{$R_{F}$}}
	&\multicolumn{1}{p{\gnumericColB}|}%
	{\gnumericPB{\raggedright}\gnumbox[l]{10k$\ohm$}}
\\
\hhline{|--|}
	 \multicolumn{1}{|p{\gnumericColA}|}%
	{\gnumericPB{\raggedright}\gnumbox[l]{$\mu$}}
	&\multicolumn{1}{p{\gnumericColB}|}%
	{\gnumericPB{\raggedright}\gnumbox[l]{1000}}
\\
\hhline{|--|}
	 \multicolumn{1}{|p{\gnumericColA}|}%
	{\gnumericPB{\raggedright}\gnumbox[l]{$f_{c}$}}
	&\multicolumn{1}{p{\gnumericColB}|}%
	{\gnumericPB{\raggedright}\gnumbox[l]{1kHz}}
\\
\hhline{|-|-|}
\end{tabular}

\ifthenelse{\isundefined{\languageshorthands}}{}{\languageshorthands{\languagename}}
\gnumericTableEnd



\caption{}
\label{table: Values_Table}
\end{table}

Now putting the appropriate values as given in Table \ref{table: Values_Table} we get:
\begin{align}
     \frac{\frac{1000}{1 + \frac{jf}{1000}}}{1 + \frac{(1)}{(10)}\frac{1000}{1 + \frac{jf}{1000}}} &= \frac{1}{\sqrt{2}}
\\
     \frac{f^2}{10^{1 2}} + \frac{101^2}{10^6} &= 2
\\
     f \approx 1.41MHz
\end{align}
\item Using ngspice verify the Closed-Loop Transfer function or $V_{o}/V_{s}$.

\solution From \ref{eq:ee18btech11011_Value_of_Vo/Vs} we know that:

\begin{align}
    \frac{V_{o}}{V_{s}} &= -9.88V/V
\end{align}

So, to verify this use the following spice file.

\begin{lstlisting}
spice/ee18btech11011/ee18btech11011.net
\end{lstlisting}

and finally to get the result use the following python code.

\begin{lstlisting}
spice/ee18btech11011/ee18btech11011_spice.py
\end{lstlisting}

Result:
\begin{lstlisting}
figs/ee18btech11011/ee18btech11011_spice_result.eps
\end{lstlisting}
\begin{lstlisting}
figs/ee18btech11011/ee18btech11011_spice_result.pdf
\end{lstlisting}

Following are the instructions to run the spice file.
\begin{lstlisting}
spice/ee18btech11011/README.md
\end{lstlisting}


\end{enumerate}
