\begin{enumerate}[label=\thesubsection.\arabic*.,ref=\thesubsection.\theenumi]
\numberwithin{equation}{enumi}
\item State the general model of a state space system specifying the dimensions of the matrices and vectors.
\\
\solution The model is given by 
\begin{align}
\label{eq:ee18btech11004_state}
\dot{\vec{x}}(t)&=\vec{A}\vec{x}(t)+\vec{B}\vec{u}(t) \\
 \vec{y}(t)&=\vec{C}\vec{x}(t)+\vec{D} \vec{u}(t)
\end{align}
%
with parameters listed in Table \ref{table:ee18btech11004}.
%
\begin{table}[!ht]
\centering
%%%%%%%%%%%%%%%%%%%%%%%%%%%%%%%%%%%%%%%%%%%%%%%%%%%%%%%%%%%%%%%%%%%%%%
%%                                                                  %%
%%  This is the header of a LaTeX2e file exported from Gnumeric.    %%
%%                                                                  %%
%%  This file can be compiled as it stands or included in another   %%
%%  LaTeX document. The table is based on the longtable package so  %%
%%  the longtable options (headers, footers...) can be set in the   %%
%%  preamble section below (see PRAMBLE).                           %%
%%                                                                  %%
%%  To include the file in another, the following two lines must be %%
%%  in the including file:                                          %%
%%        \def\inputGnumericTable{}                                 %%
%%  at the beginning of the file and:                               %%
%%        \input{name-of-this-file.tex}                             %%
%%  where the table is to be placed. Note also that the including   %%
%%  file must use the following packages for the table to be        %%
%%  rendered correctly:                                             %%
%%    \usepackage[latin1]{inputenc}                                 %%
%%    \usepackage{color}                                            %%
%%    \usepackage{array}                                            %%
%%    \usepackage{longtable}                                        %%
%%    \usepackage{calc}                                             %%
%%    \usepackage{multirow}                                         %%
%%    \usepackage{hhline}                                           %%
%%    \usepackage{ifthen}                                           %%
%%  optionally (for landscape tables embedded in another document): %%
%%    \usepackage{lscape}                                           %%
%%                                                                  %%
%%%%%%%%%%%%%%%%%%%%%%%%%%%%%%%%%%%%%%%%%%%%%%%%%%%%%%%%%%%%%%%%%%%%%%



%%  This section checks if we are begin input into another file or  %%
%%  the file will be compiled alone. First use a macro taken from   %%
%%  the TeXbook ex 7.7 (suggestion of Han-Wen Nienhuys).            %%
\def\ifundefined#1{\expandafter\ifx\csname#1\endcsname\relax}


%%  Check for the \def token for inputed files. If it is not        %%
%%  defined, the file will be processed as a standalone and the     %%
%%  preamble will be used.                                          %%
\ifundefined{inputGnumericTable}

%%  We must be able to close or not the document at the end.        %%
	\def\gnumericTableEnd{\end{document}}


%%%%%%%%%%%%%%%%%%%%%%%%%%%%%%%%%%%%%%%%%%%%%%%%%%%%%%%%%%%%%%%%%%%%%%
%%                                                                  %%
%%  This is the PREAMBLE. Change these values to get the right      %%
%%  paper size and other niceties.                                  %%
%%                                                                  %%
%%%%%%%%%%%%%%%%%%%%%%%%%%%%%%%%%%%%%%%%%%%%%%%%%%%%%%%%%%%%%%%%%%%%%%

	\documentclass[12pt%
			  %,landscape%
                    ]{report}
       \usepackage[latin1]{inputenc}
       \usepackage{fullpage}
       \usepackage{color}
       \usepackage{array}
       \usepackage{longtable}
       \usepackage{calc}
       \usepackage{multirow}
       \usepackage{hhline}
       \usepackage{ifthen}

	\begin{document}


%%  End of the preamble for the standalone. The next section is for %%
%%  documents which are included into other LaTeX2e files.          %%
\else

%%  We are not a stand alone document. For a regular table, we will %%
%%  have no preamble and only define the closing to mean nothing.   %%
    \def\gnumericTableEnd{}

%%  If we want landscape mode in an embedded document, comment out  %%
%%  the line above and uncomment the two below. The table will      %%
%%  begin on a new page and run in landscape mode.                  %%
%       \def\gnumericTableEnd{\end{landscape}}
%       \begin{landscape}


%%  End of the else clause for this file being \input.              %%
\fi

%%%%%%%%%%%%%%%%%%%%%%%%%%%%%%%%%%%%%%%%%%%%%%%%%%%%%%%%%%%%%%%%%%%%%%
%%                                                                  %%
%%  The rest is the gnumeric table, except for the closing          %%
%%  statement. Changes below will alter the table's appearance.     %%
%%                                                                  %%
%%%%%%%%%%%%%%%%%%%%%%%%%%%%%%%%%%%%%%%%%%%%%%%%%%%%%%%%%%%%%%%%%%%%%%

\providecommand{\gnumericmathit}[1]{#1} 
%%  Uncomment the next line if you would like your numbers to be in %%
%%  italics if they are italizised in the gnumeric table.           %%
%\renewcommand{\gnumericmathit}[1]{\mathit{#1}}
\providecommand{\gnumericPB}[1]%
{\let\gnumericTemp=\\#1\let\\=\gnumericTemp\hspace{0pt}}
 \ifundefined{gnumericTableWidthDefined}
        \newlength{\gnumericTableWidth}
        \newlength{\gnumericTableWidthComplete}
        \newlength{\gnumericMultiRowLength}
        \global\def\gnumericTableWidthDefined{}
 \fi
%% The following setting protects this code from babel shorthands.  %%
 \ifthenelse{\isundefined{\languageshorthands}}{}{\languageshorthands{english}}
%%  The default table format retains the relative column widths of  %%
%%  gnumeric. They can easily be changed to c, r or l. In that case %%
%%  you may want to comment out the next line and uncomment the one %%
%%  thereafter                                                      %%
\providecommand\gnumbox{\makebox[0pt]}
%%\providecommand\gnumbox[1][]{\makebox}

%% to adjust positions in multirow situations                       %%
\setlength{\bigstrutjot}{\jot}
\setlength{\extrarowheight}{\doublerulesep}

%%  The \setlongtables command keeps column widths the same across  %%
%%  pages. Simply comment out next line for varying column widths.  %%
\setlongtables

\setlength\gnumericTableWidth{%
	48pt+%
	29pt+%
	85pt+%
0pt}
\def\gumericNumCols{3}
\setlength\gnumericTableWidthComplete{\gnumericTableWidth+%
         \tabcolsep*\gumericNumCols*2+\arrayrulewidth*\gumericNumCols}
\ifthenelse{\lengthtest{\gnumericTableWidthComplete > \linewidth}}%
         {\def\gnumericScale{\ratio{\linewidth-%
                        \tabcolsep*\gumericNumCols*2-%
                        \arrayrulewidth*\gumericNumCols}%
{\gnumericTableWidth}}}%
{\def\gnumericScale{1}}

%%%%%%%%%%%%%%%%%%%%%%%%%%%%%%%%%%%%%%%%%%%%%%%%%%%%%%%%%%%%%%%%%%%%%%
%%                                                                  %%
%% The following are the widths of the various columns. We are      %%
%% defining them here because then they are easier to change.       %%
%% Depending on the cell formats we may use them more than once.    %%
%%                                                                  %%
%%%%%%%%%%%%%%%%%%%%%%%%%%%%%%%%%%%%%%%%%%%%%%%%%%%%%%%%%%%%%%%%%%%%%%

\ifthenelse{\isundefined{\gnumericColA}}{\newlength{\gnumericColA}}{}\settowidth{\gnumericColA}{\begin{tabular}{@{}p{48pt*\gnumericScale}@{}}x\end{tabular}}
\ifthenelse{\isundefined{\gnumericColB}}{\newlength{\gnumericColB}}{}\settowidth{\gnumericColB}{\begin{tabular}{@{}p{29pt*\gnumericScale}@{}}x\end{tabular}}
\ifthenelse{\isundefined{\gnumericColC}}{\newlength{\gnumericColC}}{}\settowidth{\gnumericColC}{\begin{tabular}{@{}p{85pt*\gnumericScale}@{}}x\end{tabular}}

\begin{tabular}[c]{%
	b{\gnumericColA}%
	b{\gnumericColB}%
	b{\gnumericColC}%
	}

%%%%%%%%%%%%%%%%%%%%%%%%%%%%%%%%%%%%%%%%%%%%%%%%%%%%%%%%%%%%%%%%%%%%%%
%%  The longtable options. (Caption, headers... see Goosens, p.124) %%
%	\caption{The Table Caption.}             \\	%
% \hline	% Across the top of the table.
%%  The rest of these options are table rows which are placed on    %%
%%  the first, last or every page. Use \multicolumn if you want.    %%

%%  Header for the first page.                                      %%
%	\multicolumn{3}{c}{The First Header} \\ \hline 
%	\multicolumn{1}{c}{colTag}	%Column 1
%	&\multicolumn{1}{c}{colTag}	%Column 2
%	&\multicolumn{1}{c}{colTag}	\\ \hline %Last column
%	\endfirsthead

%%  The running header definition.                                  %%
%	\hline
%	\multicolumn{3}{l}{\ldots\small\slshape continued} \\ \hline
%	\multicolumn{1}{c}{colTag}	%Column 1
%	&\multicolumn{1}{c}{colTag}	%Column 2
%	&\multicolumn{1}{c}{colTag}	\\ \hline %Last column
%	\endhead

%%  The running footer definition.                                  %%
%	\hline
%	\multicolumn{3}{r}{\small\slshape continued\ldots} \\
%	\endfoot

%%  The ending footer definition.                                   %%
%	\multicolumn{3}{c}{That's all folks} \\ \hline 
%	\endlastfoot
%%%%%%%%%%%%%%%%%%%%%%%%%%%%%%%%%%%%%%%%%%%%%%%%%%%%%%%%%%%%%%%%%%%%%%

\hhline{|-|-|-}
	 \multicolumn{1}{|p{\gnumericColA}|}%
	{\gnumericPB{\centering}\gnumbox{\textbf{Variable}}}
	&\multicolumn{1}{p{\gnumericColB}|}%
	{\gnumericPB{\centering}\gnumbox{\textbf{Size}}}
	&\multicolumn{1}{p{\gnumericColC}|}%
	{\gnumericPB{\raggedright}\textbf{Description}}
\\
\hhline{|---|}
	 \multicolumn{1}{|p{\gnumericColA}|}%
	{\gnumericPB{\centering}\gnumbox{$\vec{u}$}}
	&\multicolumn{1}{p{\gnumericColB}|}%
	{\gnumericPB{\centering}\gnumbox{$p \times 1$}}
	&\multicolumn{1}{p{\gnumericColC}|}%
	{\gnumericPB{\raggedright}input(control) vector}
\\
\hhline{|---|}
	 \multicolumn{1}{|p{\gnumericColA}|}%
	{\gnumericPB{\centering}\gnumbox{$\vec{y}$}}
	&\multicolumn{1}{p{\gnumericColB}|}%
	{\gnumericPB{\centering}\gnumbox{$q \times 1$}}
	&\multicolumn{1}{p{\gnumericColC}|}%
	{\gnumericPB{\raggedright}output vector}
\\
\hhline{|---|}
	 \multicolumn{1}{|p{\gnumericColA}|}%
	{\gnumericPB{\centering}\gnumbox{$\vec{x}$}}
	&\multicolumn{1}{p{\gnumericColB}|}%
	{\gnumericPB{\centering}\gnumbox{$n \times 1$}}
	&\multicolumn{1}{p{\gnumericColC}|}%
	{\gnumericPB{\raggedright}state vector}
\\
\hhline{|---|}
	 \multicolumn{1}{|p{\gnumericColA}|}%
	{\gnumericPB{\centering}\gnumbox{$\vec{A}$}}
	&\multicolumn{1}{p{\gnumericColB}|}%
	{\gnumericPB{\centering}\gnumbox{$n \times n$}}
	&\multicolumn{1}{p{\gnumericColC}|}%
	{\gnumericPB{\raggedright}state or system matrix}
\\
\hhline{|---|}
	 \multicolumn{1}{|p{\gnumericColA}|}%
	{\gnumericPB{\centering}\gnumbox{$\vec{B}$}}
	&\multicolumn{1}{p{\gnumericColB}|}%
	{\gnumericPB{\centering}\gnumbox{$n \times p$}}
	&\multicolumn{1}{p{\gnumericColC}|}%
	{\gnumericPB{\raggedright}input matrix}
\\
\hhline{|---|}
	 \multicolumn{1}{|p{\gnumericColA}|}%
	{\gnumericPB{\centering}\gnumbox{$\vec{C}$}}
	&\multicolumn{1}{p{\gnumericColB}|}%
	{\gnumericPB{\centering}\gnumbox{$q \times n$}}
	&\multicolumn{1}{p{\gnumericColC}|}%
	{\gnumericPB{\raggedright}output matrix}
\\
\hhline{|---|}
	 \multicolumn{1}{|p{\gnumericColA}|}%
	{\gnumericPB{\centering}\gnumbox{$\vec{D}$}}
	&\multicolumn{1}{p{\gnumericColB}|}%
	{\gnumericPB{\centering}\gnumbox{$q \times p$}}
	&\multicolumn{1}{p{\gnumericColC}|}%
	{\gnumericPB{\raggedright}feedthrough matrix}
\\
\hhline{|-|-|-|}
\end{tabular}

\ifthenelse{\isundefined{\languageshorthands}}{}{\languageshorthands{\languagename}}
\gnumericTableEnd


\caption{}
\label{table:ee18btech11004}
\end{table}
\item Find the transfer function $\vec{H}(s)$ for the general system.
\\
\solution 
Taking Laplace transform on both sides we have the following equations
\begin{align}
s\vec{I}X(s)-x(0)= \vec{A}X(s)+ \vec{B}U(s)\\
(s\vec{I}-\vec{A})X(s)= \vec{B}U(s)+ x(0)\\
X(s)={(s\vec{I}-\vec{A})^{-1}}\vec{B} U(s)+ (s\vec{I}-\vec{A})^{-1}x(0)
\label{eq:x_init}
\end{align}
and
\begin{align}
Y(s)= \vec{C}X(s)+D\vec{I}U(s)
\end{align}
Substituting from \eqref{eq:x_init} in the above,
%
\begin{multline}
Y(s)=( \vec{C}{(s\vec{I}-\vec{A})^{-1}}\vec{B}+D\vec{I}) U(s) 
\\
+ \vec{C}(s\vec{I}-\vec{A})^{-1}x(0)
\end{multline}
%
\item Find $H(s)$ for a SISO (single input single output) system.
\\
\solution
\begin{align}
\label{eq:ee18btech11004_siso}
H(s)= {\frac{Y(s)}{U(s)}}= C{(sI-A)^{-1}}B+DI
\end{align}

\item Given 
\begin{align}
H(s)&=\frac{1}{s^3+3s^2+2s+1}
\\
D&=0
\\
\vec{B}&= \myvec{0\\0\\1}
\end{align}
%
 find $\vec{A}$ and $\vec{C}$ such that the state-space realization is in {\em controllable canonical form}.
\\
\solution 
\begin{align} 
\because {\frac{Y(s)}{U(s)}}= \frac{Y(s)}{V(s)} \times \frac{V(s)}{U(s)},
\end{align}
letting
\begin{align}
 {\frac{Y(s)}{V(s)}}= 1, 
\end{align}
results in 
\begin{align}
{\frac{U(s)}{V(s)}}={s^3 + 3s^2+2s + 1}
\end{align}

giving
\begin{align}
U(s)= s^3 V(s) + 3s^2 V(s)+2sV(s) + V(s)
\end{align}

so equation 0.1.13 can be written as
\begin{align}
\myvec{sV(s)\\s^2V(s)\\s^3V(s)}
=
\myvec{0&1&0\\0&0&1\\-1&-2&-3}\myvec{V(s)\\s(s)\\s^2V(s)}
+
\myvec{0\\0\\1}  U
\end{align}
So 
\begin{align}
\vec{A}=\myvec{0&1&0\\0&0&1\\-1&-2&-3}
\end{align}

\begin{align}
Y=X_{1}(s)
=\myvec{1&0&0} \myvec{V(s)\\sV(s)\\s^2V(s)}
\end{align}
\begin{align}
\vec{C}=\myvec{1&0&0}
\end{align}

\item Obtain $\vec{A}$ and $\vec{C}$ so that the state-space realization in in {\em observable canonical form}.
\\
\solution  Given that
\begin{align}
H(s)&=\frac{1}{s^3+3s^2+2s+1}
\end{align}
\begin{align}
\frac{Y(s)}{U(s)}=\frac{1}{s^3+3s^2+2s+1} \\
Y(s) \times (s^3+3s^2+2s+1) = U(s)
\end{align}
\begin{align}
s^3Y(s)+3s^2Y(s)+2sY(s)+Y(s)=U(s)\\
s^3Y(s)=U(s)-3s^2Y(s)-2sY(s)-Y(s)\\
Y(s)=-3s^{-1}Y(s)-2s^{-2}Y(s)+s^{-3}(U(s)-Y(s))
\end{align}
\\ let $Y=aU+X_{1}$
\\ by comparing with equation 1.5.6 we get a=0 and
\begin{align}
Y=X_{1}
\end{align}
inverse laplace transform of above equation is 
\begin{align}
y=x_{1}
\end{align}
so from above equation 1.5.6 and 1.5.7
\begin{align}
X_{1}=-3s^{-1}Y(s)-2s^{-2}Y(s)+s^{-3}(U(s)-Y(s))\\
sX_{1}=-3Y(s)-2s^{-1}Y(s)+s^{-2}(U(s)-Y(s)) 
\end{align}
inverse laplace transform of above equation 
\begin{align}
\dot{x_{1}}=-3y+x_{2}
\end{align} 
where
\begin{align}
X_{2}=-2s^{-1}Y(s)+s^{-2}(U(s)-Y(s))\\
sX_{2}=-2Y(s)+s^{-1}(U(s)-Y(s))
\end{align} 
inverse laplace transform of above equation 
\begin{align}
\dot{x_{2}}=-2y+x_{3}
\end{align}
where
\begin{align}
X_{3}=s^{-1}(U(s)-Y(s))\\
sX_{3}=U(s)-Y(s)
\end{align} 
inverse laplace transform of above equation 
\begin{align}
\dot{x_{3}}=u-y
\end{align}
so we get four equations which are
\begin{align}
y=x_{1}\\
\dot{x_{1}}=-3y+x_{2}\\
\dot{x_{2}}=-2y+x_{3}\\
\dot{x_{3}}=u-y
\end{align} 
sub $ y=x_{1}$ in 1.5.19,1.5.20,1.5.21 we get
\begin{align}
 y=x_{1}\\
\dot{x_{1}}=-3x_{1}+x_{2}\\
\dot{x_{2}}=-2x_{1}+x_{3}\\
\dot{x_{3}}=u-x_{1}
\end{align} 
so above equations can be written as
\begin{align}
\myvec{\dot{x_{1}}\\\dot{x_{2}}\\\dot{x_{3}})}
=
\myvec{-3&1&0\\-2&0&1\\-1&0&0}\myvec{x_{1}\\x_{2}\\x_{3}}
+
\myvec{0\\0\\1}  U
\end{align}
So 
\begin{align}
\vec{A}=\myvec{-3&1&0\\-2&0&1\\-1&0&0}
\end{align}
\begin{align}
y=x_{1}
=\myvec{1&0&0} \myvec{x_{1}\\x_{2}\\x_{3}}
\end{align}
\begin{align}
\vec{C}=\myvec{1&0&0}
\end{align}


\item Find the eigenvaues of $\vec{A}$ and the poles of $H(s)$ using a python code.
\\
\solution The following code 
%
\begin{lstlisting}
codes/ee18btech11004.py
\end{lstlisting}
gives the necessary values.  The roots are the same as the eigenvalues.
%
\item Theoretically, show that eigenvaues of $\vec{A}$ are the poles of  $H(s)$.
\solution 
\\ as we know tthat  the characteristic equation is det(sI-A) 
\\\begin{align}
\vec{sI-A}=
\myvec{s&0&0\\0&s&0\\0&0&s}
-
\myvec{0&1&0\\0&0&1\\-1&-2&-3}
=\myvec{s&-1&0\\0&s&-1\\1&2&s+3}
\end{align}
\\therfore
\begin{align}
det(sI-A)=s(s^2+3s+2)+1(1)=s^3+3s^2+2s+1
\end{align} 
\\so from equation 1.6.2 we can see that charcteristic equation is equal to the denominator of the transefer function
\end{enumerate}

